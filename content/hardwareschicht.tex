% !TeX root = ../index.tex

\section{Bestandteile der Hardwareschicht}

Wie bereits im vorherigen Kapitel angedeutet bildet die Hardwareschicht den untersten Baustein der Systemarchitektur. In diesem Kapitel werden die Methoden vorgestellt, die während des Laborprojekts eingesetzt wurden, um mit der Hardware arbeiten zu können. Desweiteren werden die einzelnen Hardwarebestandteile aufgedrosselt und deren Funktionsweise näher beschrieben.

Beim Arbeiten mit der eingesetzten Hardware haben wir auf die browserbasierte Anwendung Wokwi gesetzt, welche über den Link 'https://wokwi.com' erreichbar ist. Die Entscheidung für die virtuelle Umgebung Wokwi wurde aus dem Grund getroffen, da das Aufbauen einer Netzwerkumgebung im Labor nicht möglich war. Die Simulation in Wokwi kann allerdings im Anschluss auf die reale Welt übertragen werden und der entwickelte Code läuft auf den entsprechenden Hardwaregeräten ohne eine notwendige Konvertierung.

Bezüglich der eingesetzten Hardware kommen bei unserer intelligenten Gartenbewässerung insgesamt zwei ESP32-Mikrokontroller des Unternehmens Espressif Systems zum Einsatz. Mithilfe der Mikrocontroller können die angeschlossenen Sensoren und Aktoren angesteuert werden. Dabei wird ein ESP32-Mikrokontroller für die Sensoren und einer für die Aktoren eingesetzt. Im folgenden werden vorerst die Sensoren und anschließend die Aktoren näher beleuchtet.

\subsection{Einsatz von Sensoren}
Als Temperatur- und Feuchtigkeitssensor setzen wir in der virtuellen Wokwi Umgebung auf einen DHT22 Sensor. Insgesamt setzen wir davon zwei Stück ein, einen für die Werte in der Luft und einen für die Werte im Boden. In der realen Welt kann der DHT22 Sensor allerdings nicht im Erdboden vergraben werden, da dieser nicht wasserresistent ist. In der Wokwi Umgebung gibt es allerdings nur eine begrenzte Auswahl an Hardwaresensorik. Im realen Anwendungsfall würde sich ein \textit{-muss ich noch herausfinden-} Sensor für die Ermittlung der Temperatur- und Feuchtigkeitswerte im Boden eignen. In Wokwi steht der zweite DHT22 Sensor somit repräsentativ für die zu erfassenden Werte im Boden. Die Werte, die der DHT22 Sensor liefert, werden zum einen in Grad Celsius und zum anderen in Prozent angegeben.

Für die Ermittlung der Helligkeit wird ein LDR Fotowiderstandsensor eingesetzt, welcher die Stärke der Sonneneinstrahlung erfasst. Bei dem Umgang bezüglich der Einheit der vom LDR Sensor gelieferten Werte haben sich ein paar Schwierigkeiten aufgetan. Der LDR Fotowiderstandsensor liefert auf dem digitalen Output Pin Werte im Bereich von 0 bis 65535. Der Wert steigt dabei, wenn die Sonneneinstrahlung schwächer wird, und sinkt, wenn es in der Umgebung heller wird. Die sensorinterne Berechnung des Wertes basiert auf dem Widerstand und der Spannung, die am Sensor anliegt. Die Spannung ändert sich je nach Sonneneinstrahlung. Auf die interne Funktionsweise des Sensors wird allerdings nicht weiter eingegangen. Die Herausforderung liegt nun dabei, den gelieferten Wert am digitalen Output Pin in die Einheit Lux umzurechnen, um damit in der weiteren Entwicklungsarbeit Logiken definieren zu können. In Abb. \ref{fig:LDR_Berechnung} ist ein MicroPython-Ausschnitt dargestellt, der den Wert des LDR Fotowiderstands in die Einheit Lux umrechnet.

\begin{figure}[h]
    \centering
    \includegraphics[width=0.8\textwidth]{LDR_Berechnung.png}
    \caption{Berechnung des Lux Werts des LDR Fotowiderstands}\label{fig:LDR_Berechnung}
\end{figure}

Die drei oben aufgeführten Sensoren werden an einen ESP32-Mikrocontroller angeschlossen, welcher mit dem hausinternen Wireless Local Area Network (WLAN) verbunden ist. Abb. \ref{fig:wokwi_sensoren} zeigt nochmals zusammenfassend einen visuellen Auszug aus Wokwi, auf dem der Anschluss der Sensorik an den ESP32-Mikrokontroller dargestellt ist. Auf der rechten Seite befinden sich die beiden DHT22 Sensoren, auf der linken Seite der LDR Fotowiderstandsensor und in der Mitte der ESP32-Mikrokontroller.

\begin{figure}[h]
    \centering
    \includegraphics[width=0.8\textwidth]{Wokwi_Sensoren.png}
    \caption{Anschluss der Sensorhardware in der virtuellen Umgebung Wokwi}\label{fig:wokwi_sensoren}
\end{figure}

\subsection{Einsatz von Aktoren}
Bezüglich der Steuerung der Markise, um die Sonneneinstrahlung zu reduzieren, verwenden wir einen Servo-Motor, der die Markise aus- und einfahren kann. Die Steuerung des Servo-Motors läuft über die MicroPython Library Servo. Über die Angabe eines Winkels bewegt sich der Servo-Motor in die gewünschte Position Position. Somit kann die Markise über die Drehung des Motors um einen bestimmten Winkel ein- bzw. ausgefahren werden.

Die Steuerung der Wasserpumpe übernimmt ein Relays, welches durch das Schalten eines neuen Stromkreises die Wasserpumpe an- bzw. ausschalten kann. In MicroPython definieren die Werte '0' und '1' des Relais, ob der Stromkreis geschlossen oder offen ist. In Wokwi haben wir zu Demonstrationszwecken eine LED an das Relay angeschlossen, um zu veranschaulichen, ob die Wasserpumpe läuft oder nicht. Im realen Anwendungsfall wird an das Relay anstatt der Wokwi-LED eine Wasserpumpe angeschlossen.

Die beiden Aktoren Servo-Motor und Relay sind an einem ESP32-Mikrokontroller angeschlossen, welcher zusammen mit dem Mikrokontroller der Sensoren mit dem hausinternen WLAN verbunden wird. Abb. \ref{fig:wokwi_aktoren} zeigt nochmals, analog zum vorherigen Unterkapitel, einen Ausschnitt aus Wokwi, auf dem die Verkabelung der Aktoren mit dem ESP32-Mikrokontroller dargestellt ist. Am oberen Bildrand befindet sich der Servo-Motor, am unteren Bildrand das Relay mit angeschlossener LED und links der ESP32-Mikrokontroller.

\begin{figure}[h]
    \centering
    \includegraphics[width=0.8\textwidth]{Wokwi_Aktoren.png}
    \caption{Anschluss der Aktorenhardware in der virtuellen Umgebung Wokwi}\label{fig:wokwi_aktoren}
\end{figure}