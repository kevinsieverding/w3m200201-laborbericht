% !TeX root = ../index.tex

\section{Einleitung}

\subsection{Motivation}
Das \gls{iot} kann in vielen Anwendungsszenarien zum Einsatz kommen, darunter auch in der Gartenbewirtschaftung. Eine effiziente Bewässerung ist von entscheidender Bedeutung, um eine optimale Pflanzenversorgung und Ressourcenschonung zu gewährleisten, insbesondere in Zeiten des Klimawandels und der zunehmenden Wasserknappheit. \gls{iot}-basierte Bewässerungssysteme versprechen eine präzise und bedarfsgerechte Bewässerungssteuerung, die sowohl den Wasserverbrauch reduziert als auch die Pflanzengesundheit und das Pflanzenwachstum fördert.

Gleichzeitig sollen durch den Einsatz von \gls{iot} Routinetätigkeiten automatisiert werden, wodurch für den Nutzer eine erhebliche Zeitersparnis ermöglicht wird.
Dadurch kann der Nutzer gar sorgenfrei in den Urlaub fahren, während das System die Bewässerung übernimmt.

\subsection{Zielsetzung}

In diesem Laborbericht wird ein \gls{iot}-Projekt vorgestellt, das die Konzeption und Implementierung einer intelligenten Gartenbewässerungsanlage zum Ziel hat. Dabei sollen verschiedene Sensoren, Aktoren und Netzwerktechnologien eingesetzt werden, um eine automatisierte, ressourceneffiziente und bedarfsgerechte Bewässerungssteuerung zu ermöglichen. Die Einbeziehung von Umgebungsparametern wie Bodenfeuchtigkeit, Temperatur, Luftfeuchtigkeit und Sonneneinstrahlung soll dabei helfen, den Wasserverbrauch zu optimieren und gleichzeitig die Pflanzenversorgung und -pflege zu verbessern.

Im Rahmen des Laborprojekts sollen zunächst die theoretischen Grundlagen und technischen Anforderungen für ein \gls{iot}-basiertes Bewässerungssystem erarbeitet werden. Darauf aufbauend soll ein Prototyp in einer simulierten Umgebung entwickelt werden, der die Integration der verschiedenen Komponenten und Technologien demonstriert.

\subsection{Aufbau}

Der Laborbericht gliedert sich in mehrere Abschnitte: Zunächst werden die technischen Anforderungen an ein \gls{iot}-basiertes Bewässerungssystem herausgearbeitet. Anschließend wird ein Prototyp in einer simulierten Umgebung entwickelt, der die Integration der verschiedenen Komponenten und Technologien demonstriert. Für die Entwicklung des Prototyps wird zuerst eine passende Systemarchitektur gewählt, deren Schichten (Hardware-, Netzwerk- und Applikationsschicht) in den darauffolgenden Kapiteln näher beleuchtet werden.