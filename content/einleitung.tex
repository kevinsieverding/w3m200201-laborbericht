% !TeX root = ../index.tex

\section{Einleitung}

\subsection{Motivation}
Das Internet der Dinge (IoT) kann in vielen Anwendungsszenarien zum Einsatz kommen, darunter auch in der Gartenbewirtschaftung. Eine effiziente Bewässerung ist von entscheidender Bedeutung, um eine optimale Pflanzenversorgung und Ressourcenschonung zu gewährleisten, insbesondere in Zeiten des Klimawandels und der zunehmenden Wasserknappheit. IoT-basierte Bewässerungssysteme versprechen eine präzise und bedarfsgerechte Bewässerungssteuerung, die sowohl den Wasserverbrauch reduziert als auch die Pflanzengesundheit und das Pflanzenwachstum fördert.

Gleichzeitig soll durch den Einsatz von IoT Routinetätigkeiten automatisiert werden, wodurch für den Nutzer eine erhelbliche Zeitersparnis ermöglicht wird.
Dadurch kann der Nutzer gar sorgenfrei in den Urlaub zu fahren, während das System die Bewässerung übernimmt.

\subsection{Zielsetzung}

In diesem Laborbericht wird ein IoT-Projekt vorgestellt, das die Konzeption und Implementierung einer intelligenten Gartenbewässerungsanlage zum Ziel hat. Dabei sollen verschiedene Sensoren, Aktuatoren und Netzwerktechnologien eingesetzt werden, um eine automatisierte, ressourceneffiziente und bedarfsgerechte Bewässerungssteuerung zu ermöglichen. Die Einbeziehung von Umgebungsparametern wie Bodenfeuchtigkeit, Temperatur, Luftfeuchtigkeit und Sonneneinstrahlung soll dabei helfen, den Wasserverbrauch zu optimieren und gleichzeitig die Pflanzenversorgung und -pflege zu verbessern.

Im Rahmen des Laborprojekts werden zunächst die theoretischen Grundlagen und technischen Anforderungen für ein IoT-basiertes Bewässerungssystem erarbeitet. Darauf aufbauend wird ein Prototyp entwickelt, der die Integration der verschiedenen Komponenten und Technologien demonstriert. 
%Schließlich sollen Experimente und Tests durchgeführt werden, um die Leistungsfähigkeit, Zuverlässigkeit und Effizienz des entwickelten Systems zu bewerten und mögliche Verbesserungen und Anpassungen zu identifizieren.

\subsection{Aufbau}

Der Laborbericht gliedert sich in mehrere Abschnitte: Nach einer Einführung in die Grundlagen des IoT und der Bewässerungstechnik werden die technischen Anforderungen und das Systemdesign der vorgeschlagenen Gartenbewässerungsanlage beschrieben. Im Anschluss daran erfolgt eine detaillierte Darstellung der eingesetzten Komponenten, der Systemarchitektur und der Kommunikationsprotokolle. Abschließend werden die Ergebnisse der durchgeführten Experimente und Tests präsentiert und diskutiert, um die Funktionsweise und das Potenzial des entwickelten IoT-Bewässerungssystems zu evaluieren und mögliche zukünftige Entwicklungsrichtungen aufzuzeigen.