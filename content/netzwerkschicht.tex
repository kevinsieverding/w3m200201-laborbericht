% !TeX root = ../index.tex

\section{Netzwerkschicht}

Die Aufgabe der Netzwerkschicht ist es die verschiedenen Mikrocontroller mit der Applikationsebene zu verbinden.
Hierzu sind die Mikrocontroller hierzu, wie bereits beschrieben, mit einem WLAN verbunden, welches ihnen Zugriff auf das öffentliche Internet gewährt.

Dort können sie sich mit einem MQTT-Broker verbinden, welcher als \gls{mom} Teil der Netzwerkschicht bildet und den einfachen Austausch von Daten zwischen den Controllern und der Anwendungsschicht ermöglicht.
Eine funktionale und übersichtliche Gestaltung der Topics ist dabei wichtig für die Verständlichkeit des Systems, weswegen die Topics des Systems einer klaren Struktur folgen:

\begin{minted}{text}
  plantzz/client
  plantzz/temperature/ground
  plantzz/temperature/air
  plantzz/humidity/ground
  plantzz/humidity/air
  plantzz/brightness
  plantzz/pump/state
  plantzz/pump/command
  plantzz/canvas/state
  plantzz/canvas/command
\end{minted}

Die Sensor-Controller schreiben ihre Messwerte zu den dazugehörigen Topics, wie zum Beispiel \texttt{plantzz/temperature/ground}, während die Aktor-Controller auf den Command-Topics hören, um Kommandos zu empfangen und ihren aktuellen status auf die State-Topics schreiben.
Zudem schreiben alle Controller ihren aktuellen Status auf das \texttt{plantzz/client} Topic und registrieren einen Last Will, damit der Status auch im Falle eines Verbindungsverlustes zum Controller aktualisiert wird.
