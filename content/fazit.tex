% !TeX root = ../index.tex

\section{Fazit}
Zusammenfassend wurde in dieser Laborarbeit eine intelligente Gartenbewässerung ausgearbeitet, welche die automatische Bewässerung und Pflege von Pflanzen mithilfe von IoT ermöglicht. Auf der Hardwareebene sorgen Sensoren wie der DHT22 Sensor und der LDR Fotowiderstand Sensor für die Lieferung der Messwerte, wie zum Beispiel die Bodenfeuchtigkeit, die Lufttemperatur oder die Lichtintensität. Aktoren, wie der Servo-Motor und das Relais, steuern das Aus- und Einfahren der Markise und das Ein- und Ausschalten der Wasserpumpe. Zwei ESP32-Mikrocontroller sind mit den Sensoren und Aktoren verbunden und veröffentlichen die Werte bzw. empfangen die Kommandos zum Steuern der Aktoren. Auf der Netzwerkebene sorgt ein MQTT-Broker für den Austausch der Nachrichten zwischen der Hardwareebene und der Applikationsebene. Über eine Topicstruktur werden die Messwerte und die Kommandos veröffentlicht bzw. empfangen. In der Applikationsschicht werden in Node-RED die Messwerte der Sensoren verarbeitet und nach einer definierten Logik die Kommandos zur Steuerung der Aktoren veröffentlicht. Über den Telegram Chat-Bot findet die Interaktion mit dem Nutzer statt, um ihm Warnungen zu senden, wenn der Boden droht zu gefrieren.

Insgesamt war das Arbeiten mit der Hardware in der virtuellen Umgebung Wokwi einfach handzuhaben. Mit dem LDR Fotowiderstand Senosor sind bezüglich der gelieferten Werte allerdings zu Beginn Schwierigkeiten aufgekommen, welche durch die Umrechnung des Werts gelöst werden konnten. Zudem stellt Wokwi nur eine begrenzte Auswahl an Sensorik und Aktorik zur Verfügung, was die Hardwareanbindung etwas erschwert hat.

Das Arbeiten mit Node-RED ist durch die grafische Oberfläche ebenfalls sehr angenehm. Der Großteil der Logik in dieser Laborarbeit konnte sehr gut mit den bereitgestellten Bausteinen umgesetzt werden, teilweise mit einer etwas umständlichen Verkettungen von mehreren Bausteinen hintereinander, um die gewünschte Logik zu implementieren. Node-RED kommt allerdings auch bei komplexeren Logiken an seine Grenzen.

Als Ausblick kann das Gartenbewässerungssystem um zusätzliche Funktionen erweitert werden. Chat-Bot Funktionen, wie das Senden von manuellen Kommandos zur Steuerung der Wasserpumpe und der Markise, sind in Node-RED mit geringem Aufwand umsetzbar und erweitern den Funktionsumfang des Systems nochmals.