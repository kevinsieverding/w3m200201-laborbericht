%% !TeX root = ../index.tex

\section{Quellcode}

\begin{listing}[!ht]
\begin{minted}{python3}
from machine import ADC, Pin
import dht
import time
import ujson
import network
from umqtt.simple import MQTTClient 

# MQTT server parameters
MQTT_CLIENT_ID = "plantzz-iot-client"
MQTT_BROKER    = "xxx.xxx.de"
MQTT_USER      = "xxx"
MQTT_PASSWORD  = "xxx"

# Topic declaration
MQTT_TOPIC_TEMP_GROUND = "plantzz/temperature/ground"
MQTT_TOPIC_TEMP_AIR = "plantzz/temperature/air"
MQTT_TOPIC_HUMIDITY_GROUND = "plantzz/humidity/ground"
MQTT_TOPIC_HUMIDITY_AIR = "plantzz/humidity/air"
MQTT_TOPIC_BRIGHTNESS = "plantzz/brightness"

def connect_wifi():
    print("Connecting to WiFi", end="")
    sta_if = network.WLAN(network.STA_IF)
    sta_if.active(True)
    sta_if.connect('Wokwi-GUEST', '')
    while not sta_if.isconnected():
    print(".", end="")
    time.sleep(0.1)
    print("WiFi Connected!")

connect_wifi()

print("Connecting to MQTT server... ", end="")
mqtt_client = MQTTClient(MQTT_CLIENT_ID, MQTT_BROKER, user=MQTT_USER, password=MQTT_PASSWORD, keepalive=20)
mqtt_client.set_last_will("plantzz/espsensors/status", "offline", retain=True)

mqtt_client.connect()
mqtt_client.publish('plantzz/espsensors/status', 'online', retain=True)
print("MQTT Connected!")
\end{minted}
\caption{Python-Implementierung des ESP der Sensoren (Part 1)}
\label{list:wokwi_sensoren1}
\end{listing}

\begin{listing}[!ht]
\begin{minted}{python3}
# Attributes of LDR - default values
LDR_RL10 = 50
LDR_GAMMA = 0.7

# Define input pins
sensor_dht_ground = dht.DHT22(Pin(5))
sensor_dht_air = dht.DHT22(Pin(18))
# we have to use some ADC1 pin (32-39), because WiFi and ADC2 doesn't work together
ldr = ADC(Pin(32, Pin.IN))

def publish_sensor_value(sensor_value, topic_name):
    message = ujson.dumps({
        "value": sensor_value,
    })
    print("Reporting to MQTT topic {}: {}".format(topic_name, message))
    mqtt_client.publish(topic_name, message)

while True:
    # Get DHT values
    sensor_dht_ground.measure()
    sensor_dht_air.measure()

    # Read analog value from LDR and convert to digital value
    ldr_digital_value = ldr.read_u16()
    # Convert digital value to lux
    ldr_voltage = ldr_digital_value / 65535 * 5
    ldr_resistance = 2000 * ldr_voltage / (1 - ldr_voltage / 5)
    brightness_value = round(pow(LDR_RL10 * 1e3 * pow(10, LDR_GAMMA) / ldr_resistance, (1 / LDR_GAMMA)))
    
    sensors = {
        MQTT_TOPIC_TEMP_GROUND: sensor_dht_ground.temperature(),
        MQTT_TOPIC_HUMIDITY_GROUND: sensor_dht_ground.humidity(),
        MQTT_TOPIC_TEMP_AIR: sensor_dht_air.temperature(),
        MQTT_TOPIC_HUMIDITY_AIR: sensor_dht_air.humidity(),
        MQTT_TOPIC_BRIGHTNESS: brightness_value,
    }

    # Publish sensor values
    for key in sensors:
        publish_sensor_value(sensors[key], key)
    
    time.sleep(5)
\end{minted}
\caption{Python-Implementierung des ESP der Sensoren (Part 2)}
\label{list:wokwi_sensoren2}
\end{listing}

\begin{listing}[!ht]
\begin{minted}{python3}
from machine import Pin,PWM
import time
import ujson
import network
from umqtt.simple import MQTTClient 
from servo import Servo

# MQTT Server Parameters
MQTT_CLIENT_ID = "plantzz-iot-client-actuators"
MQTT_BROKER    = "xxx.xx.de"
MQTT_USER      = "xxx"
MQTT_PASSWORD  = "xxx"

MQTT_TOPIC_PUMP_STATE = "plantzz/pump/state"
MQTT_TOPIC_PUMP_COMMAND = "plantzz/pump/command"
MQTT_TOPIC_CANVAS_STATE = "plantzz/canvas/state"
MQTT_TOPIC_CANVAS_COMMAND = "plantzz/canvas/command"

def connect_wifi():
    print("Connecting to WiFi", end="")
    sta_if = network.WLAN(network.STA_IF)
    sta_if.active(True)
    sta_if.connect('Wokwi-GUEST', '')
    while not sta_if.isconnected():
    print(".", end="")
    time.sleep(0.1)
    print("WiFi Connected!")

connect_wifi()

print("Connecting to MQTT server... ", end="")
mqtt_client = MQTTClient(MQTT_CLIENT_ID, MQTT_BROKER, user=MQTT_USER, password=MQTT_PASSWORD, keepalive=100)
mqtt_client.set_last_will("plantzz/espactuators/status", "offline", retain=True)
\end{minted}
\caption{Python-Implementierung des ESP der Aktoren (Part 1)}
\label{list:wokwi_aktoren1}
\end{listing}

\begin{listing}[!ht]
\begin{minted}{python3}


relay = Pin(15, Pin.OUT)

def publish_actuator_state(topic_name, actuator_state):
    message = ujson.dumps({
        "state": actuator_state,
    })
    print("Reporting to MQTT topic {}: {}".format(topic_name, message))
    mqtt_client.publish(topic_name, message)
    
def control_pump(command):
    print(command)
    if (command == 'activate'):
        relay.value(1)
        publish_actuator_state(MQTT_TOPIC_PUMP_STATE, 'active')
    elif (command == 'deactivate'):
        relay.value(0)
        publish_actuator_state(MQTT_TOPIC_PUMP_STATE, 'inactive')

motor=Servo(pin=18)

def control_canvas(command):
    if (command == 'extend'):
        motor.move(180)
        publish_actuator_state(MQTT_TOPIC_CANVAS_STATE, 'extended')
    elif (command == 'retract'):
        motor.move(0)
        publish_actuator_state(MQTT_TOPIC_CANVAS_STATE, 'retracted')

def sub(topic, msg):
    print('received message %s on topic %s' % (msg, topic))
    obj =  ujson.loads(msg)
    command = obj['command']
    topic = topic.decode("UTF-8")
    if (MQTT_TOPIC_PUMP_COMMAND == topic):
      control_pump(command)
    elif (MQTT_TOPIC_CANVAS_COMMAND == topic):
       control_canvas(command)
    

mqtt_client.set_callback(sub)

mqtt_client.connect()
mqtt_client.publish('espactuators/status', 'online', retain=True)
\end{minted}
\caption{Python-Implementierung des ESP der Aktoren (Part 2)}
\label{list:wokwi_aktoren2}
\end{listing}



\begin{listing}[!ht]
\begin{minted}{python3}
# this function simulates mqtt messages to test the actuators
def test_messages():
    sub(b"plantzz/pump/command", b'{"command":"activate"}')
    sub(b"plantzz/canvas/command", b'{"command":"extend"}')
    time.sleep(4)
    sub(b"plantzz/pump/command", b'{"command":"deactivate"}')
    sub(b"plantzz/canvas/command", b'{"command":"retract"}')
test_messages()

mqtt_client.subscribe("plantzz/pump/command")
mqtt_client.subscribe("plantzz/canvas/command")

print("MQTT Connected!")

while True:
    mqtt_client.wait_msg()
    time.sleep(1)

mqtt_client.disconnect()
\end{minted}
\caption{Python-Implementierung des ESP der Aktoren (Part 3)}
\label{list:wokwi_aktoren3}
\end{listing}

\begin{listing}[!ht]
\begin{minted}{python3}
node-red:
influxdb-config:
influxdb-data:
grafana_storage:

services:
    node-red:
        image: nodered/node-red
        container_name: node-red
        restart: unless-stopped
        volumes:
        - node-red:/data
        ports:
        - 80:1880
    influxdb:
        image: influxdb
        container_name: influxdb
        restart: unless-stopped
        volumes:
        - influxdb-config:/etc/influxdb2
        - influxdb-data:/var/lib/influxdb2
        ports:
        - 8086:8086
        env_file: .env-influxdb
    grafana:
        image: grafana/grafana-enterprise
        container_name: grafana
        restart: unless-stopped
        volumes:
        - grafana_storage:/var/lib/grafana
        ports:
        - 3000:3000
\end{minted}
\caption{Definition der Container in docker-compose.yml}
\label{list:docker}
\end{listing}  